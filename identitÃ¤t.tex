\documentclass[a4paper,12pt,titlepage]{scrartcl}

%Pakete
%\usepackage[left=3cm,right=2cm,top=2cm,bottom=3cm]{geometry} %Seitenränder
\usepackage[utf8]{inputenc} % ermöglicht die direkte Eingabe der Umlaute 
\usepackage [german]{babel} %Spracheinstellungen

\usepackage {csquotes} % Anführungszeichen nach dem Stil \enquote{ich bin zitiert.}
\usepackage{subscript} % erlaubt Tiefstellen von Zahlen und Text
\sloppy %macht ungefähren Blocksatz, wenn nichts anderes an Trennhilfen was nützt

% Hurenkinder und Schusterjungen verhindern
\clubpenalty10000
\widowpenalty10000
\displaywidowpenalty=10000

\usepackage[final]{graphicx}

\usepackage [
citestyle = authoryear,
bibstyle = philosophy-classic,% Zitierstil
isbn=false,                % ISBN nicht anzeigen, %gleiches geht mit nahezu allen anderen Feldern
doi=false,
pagetracker=true,          % ebd. bei wiederholten ngaben (false=ausgeschaltet, page=Seite, spread=Doppelseite, true=automatisch)
%ümaxbibnames=3,            % maximale Namen, die im Literaturverzeichnis angezeigt werden 
maxcitenames=3,            % maximale Namen, die im Text angezeigt werden, ab 4 wird u.a. nach den ersten Autor angezeigt
autocite=inline,           % regelt Aussehen für \autocite (inline=\parancite)
block=space,               % kleiner horizontaler Platz zwischen den Feldern
backref=false,              % Seiten anzeigen, auf denen die Referenz vorkommt
backrefstyle=three+,       % fasst Seiten zusammen, z.B. S. 2f, 6ff, 7-10
date=short                % Datumsformat
]{biblatex}
\newcommand{\todo}[1]{\marginpar{#1}}
\bibliography{sources}

\DeclareBibliographyDriver{report}{%
  \usebibmacro{bibindex}%
  \usebibmacro{begentry}%
  \usebibmacro{author}%
  \setunit{\labelnamepunct}\newblock
  \usebibmacro{title}%
  \newunit
  \printlist{language}%
  \newunit\newblock
  \usebibmacro{byauthor}%
  \newunit\newblock
%  \printfield{type}%
%  \setunit*{\addspace}%
  \printfield{number}%
  \newunit\newblock
  \printfield{version}%
  \newunit
  \printfield{note}%
  \newunit\newblock
  \usebibmacro{institution+location+date}%
  \newunit\newblock
  \usebibmacro{chapter+pages}%
  \newunit
  \printfield{pagetotal}%
  \newunit\newblock
  \iftoggle{bbx:isbn}
    {\printfield{isrn}}
    {}%
  \newunit\newblock
  \usebibmacro{doi+eprint+url}%
  \newunit\newblock
 \usebibmacro{addendum+pubstate}%
  \setunit{\bibpagerefpunct}\newblock
  \usebibmacro{pageref}%
  \usebibmacro{finentry}}
\usepackage[colorlinks=true,linkcolor=black,citecolor=black,urlcolor=black,breaklinks=true]{hyperref}
%\usepackage[style=authortitle-icomp]{biblatex}

%\bibliography{Pfad/zur/Bibliographie-Datei/Dateiname} 


\title{Automatisierung und Identitätsdiffusion}
\subtitle{Wie Mechanisierung und Automatisierung unsere Identität beeinflussen}
\author{Sebastian Mai}
\date{\today} % sollte man u.U. anpassen ;)


\begin{document}
\maketitle
\tableofcontents
\section {Mechanisierung, Automatisierung und Identität}

Unter dem Umgangssprachlich gebrauchten Begriff \enquote{Automatisierung} verbergen sich eigentlich zwei voneinander Abgrenzbare Begriffe: Mechanisierung und Automatisierung (im eigentlichen Sinne).

\subsection {Mechanisierung}

Als Mechanisierung Bezeichen ich in dieser Arbeit den Einsatz von Techniken, die menschliche Kraft unterstützen oder ersetzen, und so den Aufwand für unsere Handlungen senken, oder deren Etrag steigern.
Das durchführen der eigentlichen Handlung ist dabei aber immer noch notwendig.

Ein Rechenschieber und Rechentabellen beispielsweise machen es möglich, dass ich mit kleinerem Aufwand Berechnungen durchführen kann.
Der Einsatz des Rechenschiebers ist also als Mechanisierung zu verstehen.

\subsection {Automatisierung}

Automatisierung wird für die Industrie definiert als: \enquote{Das Ausrüsten einer Einrichtung, so dass sie ganz oder teilweise ohne Mitwirkung des Menschen bestimmungsgemäß arbeitet.}\parencite{din19233}.

Die Automatisierung kann als Weiterführung der Mechanisierung verstanden werden\parencite{ulrich}, ist allerdings dadurch ausgezeichnet, dass der Mensch nicht mehr Steuernd in Prozesse eingreifen muss und alle Zwischenschritte durch die Maschine erledigt werden.
Wichtig für die Abgrenzung zur Mechanisierung ist, dass die ursprüngliche Handlung im automatisierten Vorgang gar nicht mehr durchgeführt wird – während durch eine Mechanisierung nur verändert wird.


Führen wir das Beispiel des Rechnens weiter, so ist die Automatisierung des Rechnens der Einsatz eines Computers, der seine Eingaben selbständig abruft und auch selbstständig ein Ergebnis errechnet.

\subsection {Identität}
Identität verstehe ich zunächst als die Antwort auf die Frage: \enquote{Wer bin ich?}.

Die Fragen die dabei mitschwingen sind \enquote{Wer war ich?} und \enquote{Was sind meine Ziele?} – also eine Perspektive die einerseits unsere Vergangenheit berücksichtigt und andererseits Möglichkeitsräume einbezieht.
Neben dieser zeitlichen Perspektive werden bei der Identitätsbildung Verknüpfungsarbeit geleistet zwischen lebensweltlichen und inhaltlichen Perspektiven.
Das aushandeln der Relevanz der einzelnen Tatsachen ist zentraler Bestndteil der Identitätsarbeit.
Die Konflikte, die sich dabei aus den verschiedenen Identitätsbestandteilen \footnote{self, me, other, ideal ...} ergeben, sind Quelle der Dynamik des Identitätsprozesses \parencite{keupp}.

\subsection {Das Verhältnis von Mechanisierung und Automatisierung zur Identität}

Diese Möglichkeitsräume werden stark beeinflusst von Mechanisierung und Automatisierung.
Einerseits eröffnen sich neue Möglichkeiten, da Mechanisierung schlichtweg unsere Effizienz und Effektivität steigern, andererseits verschwinden auch Möglichkeiten für unser Handeln, zum Beispiel weil alte Berufe nicht mehr ausgeübt werden.

Die Abgrenzung zwischen beiden Prozessen der Mechanisierung und Automatisierung ist oft nicht eindeutig und der Übergang von Mechanisierung zur Automatisierung oft fließend, allerdings ist dieser Übergang wichtig für die Identität derer die davon Betroffen sind.
So kann ich mich als Mensch mit Rechenschieber, Zettel und Stift noch \enquote{Computer}\footnote{\enquote{Computer} war früher tatsächlich eine Berufsbezeichnung. Das wir heute  eine Maschine, aber niemals einen Menschen so bezeichnen würden zeigt, wie überholt dieses Identitätsangebot ist.} nennen heutzutage wo das eigentliche Rechen aber vollständig automatisch abläuft geht das nicht mehr.
Ich bin vielleicht \enquote{Systemadministrator} oder \enquote{Programmierer}, aber die eigentliche Tätigkeit des Rechnens gehört jetz nicht mehr zu meiner Tätigkeit – und ist damit auch nicht mehr identitätsstiftend.

Den technischen Prozess zu überwachen ist in gewissen weise eine logische Nachfolge der eigentlichen Tätigkeit, allerdings kann es für die Identitätsbildung nicht das leisten, was die ursprüngliche Tätigkeit leisten kann.
Die Arbeit wird abstrakter und bekommt meist einen weiteren Themenumfang als es ursprünglich der Fall war \parencite{ulrich} – Automatisiert und Mechanisiert werden zunächst mechanisch einfache Vorgänge von niedrigem Komplexitätsgrad, die häufig wiederholt werden.
Zusätzlich zur Verlagerung des Arbeitsfeldes ist auch der durch die Automatisierung erreichte Rationalisierungseffekt wichtig für Identitätskonzepte.
Von den 100 Bauern die vor 100 Jahren noch auf einem Bauernhof gearbeitet haben, sind heute gerade noch anderthalb Stellen für sehr gut ausgebildete Landwirte übriggeblieben \parencite{arbeitsfrei} – die übrigen 98,5 Menschen verlieren die Identitätsstiftenden tätigkeiten auf dem Bauernhof komplett.

Da heutzutage alle Lebensbereiche in der ein oder anderen Weise von Automatisierung betroffen sind, ist anzunehmen, dass sich usere Identitätskonzepte zwangsläufig ändern müssen.


\section{Geschichte der Automatisierung}

Mechanisierung und Automatisierung gibt es seit der Mensch Werkzeuge benutzt, aber enorme Folgen für die Sozialstruktur beginnen in der Industriellen Revolution.
Wenn wir \enquote{Müller} sagen, dann haben wir meist ein romantisches Bild in unserer Vorstellung, von einer staubigen alten Mühle mit Wind- oder Wasserrad in dem einige Müller ihr Tagewerk tun.
In einer Modernen Mühle dagegen arbeiten kaum Menschen \footnote{Vermutlich haben die Museumsmühlen insgesamt mehr Mitarbeiter als die Industriemühlen in denen unser Mehl hergestellt wird}, das Personal dort wird zum Überwachen und Instandhalten der Industrieanlagen gebraucht und nicht mehr zum mahlen von Getreide.
Das Bild des \enquote{Müllers} passt nicht einmal im Ansatz in die Postmoderne, und doch ist es das Identitätsbild, das wir im Kopf haben.

Welchen Kriesen sich diese Identität gegenüber sieht und wie sich unser Konzept von Identität seitdem geändert hat, möchte ich in diesem Kapitel erforschen.

%\todo{technologien der sozialen Sättigung}

\subsection{Ausgangslage™}

Vor der Industriellen Revolution gab es nur eine Agrargesellschaft, die Frage \enquote{Wer bin ich?} lies sich relativ einfach beantworten mit \enquote{Ich bin Bauer [, Müller, Pfarrer, Schmied \footnote{Darin dürften sich die Rollenbilder der Zeit auch fast schon erschöpfen}] aus Kleinbuxtehude}.


\subsection{Industrielle Revolution}

Zum Beginn der Industriellen Revolution haben wir es vor allem mit Mechanisierung zu tun.
Mit Entwicklung der Dampfmaschine wird dem Menschen (körperliche) Arbeit abgenommen, die jetzt von Maschinen erledigt wird \parencite{landes}.
Diese Revolution konnte überhaupt erst vonstatten gehen, da in der Landwirtschaft genügend Überschüsse erwirtschaftet wurden konnten, um Fabrikarbeiter zu ernähren. Diese Überschüsse wurden nicht aus purem Fleiß erarbeitet, sondern sind Folge einer Agrarrevolution, deren Neuerungen man ebenfalls als Mechanisierung auffassen kann \parencite{weissenborn, prass}.

Schon in der industriellen Revolution gab es eine heftige Gegenbewegung, gegen die um sich greifende Mechanisierung: Die Maschinenstürmer.
Beispielsweise \enquote{zerstörten Wolltuchspinner und -weber, Kattundrucker und Tuchscherer neue [Textil-]Maschinen}\parencite[44]{spehr}.
Man kann schon an diesem Beispiel erkennen, dass es eine konkrete Identitätszuschreibung der handelnden Maschinenstürmer gibt (ihr Beruf in der Textilverarbeitung) und das sich diese Identität in der damaligen Form nicht aufrechterhalten ließ.


Diese Welle der Mechanisierung hört mit der Industriellen Revolution nicht auf.
Eine weiter Entwicklung ist zum Beispiel die der Fließbandarbeit.
Interessanterweise führt hier die Mechanisierung des Transportes der Baugruppen durch die Fabrik nicht nur zu einer Arbeitsersparnis bei den jeweiligen Fertigungsprozessen, sondern auch die Komplexität der einzelnen Arbeitsschritte ist letztendlich geringer als vorher – was eine weiter Mechanisierung und Automatisierung dieser Arbeitsschritte begünstigt.

\subsection {Digitale Revolution, Computerisierung}
Mit der Erfindung und Miniaturisierung der Computer geht die Automatisierung in eine neue Phase ein. Wo früher fast nur Produktionsprozzesse automatisiert wurden, erfasst die Automatisierung heute jeden Lebensbereich.
Wo vorher nur einfache Maschinen autonom agieren konnten machen neue Sensoren und deren rechnergestützte Auswertung Regelkreise möglich, die vorher wesentlich schwieriger zu realisieren waren.
Diese Regelkreise wiederum sind wichtiger Bestandteil autonom agierender Systeme \parencite{ulrich}.


Sogar unsere Kommunikation ist von Automatisierung Betroffen – das \enquote{Das wollte ich gar nicht schreiben, das war die Autorkorrektur auf meinem Telefon} heutzutage ein völlig normaler Narrativ ist zeigt, welches Ausmaß Automatisierung heute annimmt; es ist auch nicht davon auszugehen, dass das Automatisierungslevel nicht noch sehr viel weiter steigen wird \parencite{arbeitsfrei}.

Wie weit die Automatisierung in Bereiche vorgedrungen ist, die man vor wenigen Jahren noch für komplett Automatisierungssicher hielt sieht man, wenn man sich ansieht wie Algorithmen einfache Berichte verfassen. Bereits 2005 wurden automatisch generierte Artikel zu Fachkonferenzen angenommen. Die Arbeiten die damals generiert wurden hatten noch keinen Sinnvollen Inhalt \parencite{scigen}.
Heute allerdings lassen sich auch komplexere Texte wie den Kommentar zu einem Fußballspiel oder die Zusammenfassung von Unternehmensdaten erstellen\parencite{bou}.

Wo die Reise hingehen könnte ist auch schon in Utopie und Dystopie festgehalten.
Einerseits könnten wir irgendwann in einem \enquote{Digitalen Athen} leben, so das jeder wie ein Bürger des Antiken Athen frei seinen Interessen, wie Philosophie, Naturwissenschaften und Politik nachgehen kann, während nicht wie im Alten Athen Sklaven, sondern Maschinen für uns Arbeiten.
Andererseits wird es immer Aufgaben geben, die von Maschinen nicht erledigt werden können.
So ist es schon heute der Fall, dass Menschen Aufgaben abarbeiten, die ihnen von einer Maschine vorgegeben werden – dass könnte zum Beispiel so aussehen, dass ein Lagerarbeiter einem auf einer VR-Brille eingeblendeten Weg folgt und angezeigte Gegenstände durch das Lager transportiert, weil das Greifen und einsortieren der Gegenstände für Maschinen sehr schwierig zu realisieren ist\footnote{Ähnlich dazu konnte man sich bei World of Warcraft früher ein Plugin installieren, das vorgegeben hat wohin die Spielfigur als nächstes gesteuert werden sollte um möglichst effizient das Maximallevel zu erreichen. Genauso sagen uns unsere Navigationssysteme mittlerweile mit beruhigender Stimme an, wo wir als nächstes abbiegen sollen.} – ein Leben im Stählernen Gehäuse.

\subsection{Identität}

Im Mittelalter, also vor Beginn der großen Mechanisierungswelle der Industriellen Revolution, waren die Menschen im wesentlichen von Tradition geleitet \parencite{riessman}.
Die Identität des einzelnen ist a priori festgelegt, meist von Geburt an \parencite{rosa}.

Nach Riesman ist zur Zeit der Industriellen Revolution der Idealtyp des Innengeleiteten Menschen vorherrschend. Der Innengeleitet Mensch wird durch bestimmte Werte geprägt denen er dann wie einer Kompassnadel folgt.
Die Identität wird als nicht mehr zur Geburt, sondern im Laufe des Lebens festgelegt, aber dann nicht mehr verändert.

Riesman begründet den Wandel der Identitätskonzepte durch das Bevölkerungs"-wachstum und die daraus resultierende Ressourcenknappheit, das wohl nicht zuletzt eine Folge der Agrarrevolution sein dürfte.
Versucht man die Änderungen komplett aus der Automatisierung und Mechanisierung zu begründen, so kann man anführen, dass die tradierte Verhaltensweise schlichtweg nicht mehr anwendbar waren. \todo{bullshit?}

Mit zunehmender Rationalisierung in den Produktionsprozessen und einem einhalten des Bevölkerungswachstumes herrscht nun nicht mehr das Idealbild des Innengeleiteten, sondern das des Außengeleiteten Menschen vor.
Die Identität wird nicht mehr zu einem speziellen Zeitpunkt festgelegt, sondern in jeder Situation neu ausgehandelt.

\section{Postmoderne Identitätskonzepte und Automatisierung}

Da ich längst nicht der erste bin, der sich mit Identität in der Postmoderne Beschäftigt, möchte ich nicht von Null auf anfangen ein neues, auf Automatisierung beruhendes Identitätskonzept zu beschreiben, sondern bestehende Identitätskonzepte der Postmoderne aus dem Blickwinkel der Automatisierung und Mechanisierung beschreiben.

Als Effekt dieser Veränderung lässt sich vorwegnehmen: Die Kontinuität der einzelnen Identitäten schwindet immer mehr und obwohl die Möglichkeiten für den einzelnen de facto immer mehr werden, sinkt das Kontingenzbewusstsein.
Außerdem lässt sich feststellen, dass Individualisierung schwieriger wird, da zwar die Ausdrucksmöglichkeiten für den Einzelnen immer größer werden, aber es mit zuhnehmender Pluralisierung der Lebensformen auch immer schwieriger wird, aus der Masse herauszustechen.

\todo{Kontingenzbewusstsein, Individualisierung, Kontinuität}

\subsection{Beschleunigung und Situative Identität}

Hartmuth Rosa führt als westentliche Ursache der Kriese von IdentitätskonzeptenVeränderungen in der Temporalstruktur an \parencite{rosa}.
Es ist durchaus plausibel, dass diese Veränderungen zumindest Teilweise in der vortschreitenden Mechanisierung und Automatisierung begründet liegen.
Diese Beschleunigung spiegelt sich in drei Aspekten wieder: Technische Beschleunigung, Beschleunigung des individuellen Lebenstempos und des sozialen Wandels.

Die Technische Beschleunigung von der Rosa spricht ist genau die, die aus Mechanisierung und Automatisierung hervorgeht – eine Reise die früher Tage gedauert hätte, dauert heute nur wenige Stunden, wo man früher einen Brief geschrieben hätte, schreibt man heute eine Mail.
Nicht unabhängig davon ist die Beschleunigung des individuellen Lebenstempos.
Paradoxer Weise führt die Einsparung von Zeit nicht dazu, dass uns mehr Zeit zur Verfügung steht – ganz im Gegenteil: Die Verkürzung einzelner Handlungsepisoden führt zu einer weiteren Steigerung des Lebenstempos, da mehr Handlungen in kürzerer Zeit ausgeführt werden\footnote{Genau das wird auch in der Debatte um eine \enquote{Automatisierungsdividende} thematisiert, also der Überlegung die versucht die durch Automatisierung freigewordenen Ressourcen für einzelne Personen zu nutzen, anstatt weiter nach Vollbeschäftigung mit langen Arbeitszeiten zu streben\parencite{faz}.}.
Es die einzelnen Handlungsepisoden sind also verkürzt, gleichzeitig steigt die Handlungsdichte.

Die dritte Art der Beschleunigung – die des Sozialen Wandels – lässt sich nicht auf Automatisierung oder Mechanisierung zurückführen.
Sich verändernde soziale Praktiken, Werte oder Traditionen ändern sich zwar möglicherweise weil sich unsere Umgebung durch technische Entwicklung ändert, die enstehende Struktur ist aber sicher nicht auf diese abstrakte Art von Veränderung zurückzuführen.
\todo{den nächsten satz nochmal neu bitte}
In den Änderungen der Sozialen Strukturen, liegt auch ein ganz deutliches Problem der These, dass Automatisierung die Struktur von Identität beiträgt, denn Soziale Änderungen lassen sich kaum auf Automatisierung zurückführen\footnote{Technologie gibt zwar auch Sozialstrukturen vor (beispielsweise auf Twitter, wo durch die Festlegung wer wann mit wem kommunuzieren darf) allerindgs sind diese Neuerungen nicht strukturell von Automatisierungseigenschaften abhängig}.

Wie wirkt sich nun die Beschleunigung auf unsere Identität aus? Aufgrund der Vielzahl an Optionen, die unsere Indentitätsvorstellungen Betreffen kommt es zu einer Artikulationsnot, die nicht zuletzt in einem gesteigertem Kontingenzbewusstsein begründet liegt.
All die zum Teil technischen Möglichkeiten alles mögliche zu tun – und am nächsten Tag vielleicht etwas komplett neues zu tun, berauben uns des Vokabulars unsere Identitätsvorstellungen in Worte zu fassen und unsere Entscheidungen in Abgrenzung zu all den anderen Identitätsangeboten zu begründen.
Wo früher die Aussage \enquote{Ich möchte Mathematiker werden} eine klare Karriere vorgegeben hätte, wird man heute fragen, welche Richtung der Mathematik denn gemeint sei und auf welcher Laufbahn man dort hinzukommen gedenkt.
Und das alles in der Gewissheit, dass man mit abgeschlossenem Mathemitkstudium vielleicht am Ende trotzdem als Programmierer arbeitet.

Neben dieser Artikulationsnot gibt es allerdings einen Zwang zur Selbstthematisierung – denn ohne uns selbst ständig zum Thema zu machen können wir uns in den gewachsenen Möglichkeitsräumen nicht mehr zurechtfinden.
Auch die gestiegene Geschwindigkeit und zeitliche Dichte der Kommunikation trägt dazu bei.

Genau diese Änderung im Kommunikationsverhalten lässt sich aber wiederum auf allgemeine Eigenschaften der Mechanisierung zurückführen.
Zu aller erst werden die wenig komplexen Kommunikationsformen durch technische Werkzeuge ersetzt: Massenmedien wie Zeitungen, Radio oder die Kommunikation per Post. Jetzt allerdings sehen wir neue Werkzeuge mit denen sich komplexere Kommunikationsformen abbilden lassen: Telefonkonferenzen, Gruppenchats, Mailinglisten, Webforen – alles Kommunikationsformen bei denen es nichtmehr zwei Kommunikationspartner oder einen Autor der im wesentlichen unidirektional Kommuniziert gibt, sondern eine multidirektionale Kommunikation in einem Netzwerk zulässt.

\subsection{Flexible Identität}


\subsection{Narrative Identität}

Kommen wir noch einmal zurück zur Artikulationsnot. Auch ein Teil dieser Artikulationsnot dürfe seinen Ursprung in der durch Automatisierung und Mechanisierung gestiegenen Komplexität haben.
Ein Tuchwebermeister vor der Industriellen Revolution beantwortet die Frage \enquote{Was machst du eigentlich?} vermutlich relativ einfach und nachvollziehbar.
Der Besitzer einer Textilfabrik muss da schon weiter ausholen und komplizierte Maschinen beschreiben.
Unsere Narrative sind also ganz wesentlich beeinflusst von der gestiegenen Beschreibungskomplexität unserer Tätigkeiten.

\section {Automatisierte Identität}

Mitlerweile sind nicht nur die Möglichkeitsräume und Angebote unserer Identität von Automatisierung betroffen.
Das unsere Identität ist selbst durch Automatisierung betroffen ist, soll das folgende Beispiel zeigen.

\subsection {Mechanisierung und Automatisierung der Nationalidentität}

Stellen wir uns einmal vor wir stehen an der Grenze und möchten in ein anderes Land einreisen.
Normalerweise sehen wir dort einen Grenzbeamten, der wissen möchte wer wir sind.
Die Technik mit dem wir diesen Prozess sehr einfach machen ist der Pass, man könnte sagen das vorzeigen des Passes ist eine mechanisierte Form unserer Identifizierung.
Allerdings ist heute schon absehbar, dass das vorzeigen des Passes und die Arbeit des Grenzbeamten der unsere Identität feststellt bald der Vergangenheit angehören könnte.
Die Technologie der Biometrie, also der automatischen Identitätsfeststellung übernimmt diese Aufgabe \parencite{knaut}.
Man es ist also auch durchaus möglich Identitätsarbeit durch Technologien zu automatisieren.
Wie wird sich also unsere Identität verändern, wenn wir Teilaspekte davon nicht mehr durch Narrationsarbeit kommunizieren müssen?
Verlieren wir möglicherweise auf eine neue Art die Kontrolle über unsere Identitäten?
In einer großen Geste den eigenen Pass zu verbrennen mit der Deklaration man sei jetzt \enquote{Freier Weltenbürger} wird schließlich bald nicht mehr möglich sein.
Andererseits haben alle Biometrischen Verfahren auch schwächen und ein neues Phänomen wird interessant: Identitätsdiebstahl. \todo{quelle identitätsdiebstahl}
Schließlich geben wir nicht nur auf mechanisiertem, oder gar automatisiertem Wege unsere Identität für die Grenzbeamten preis – wir \enquote{identifizieren} uns auch gegenüber unserer Bank oder unseren Rechnern.

\subsection {Automatisierung der Identitätsarbeit im Internet}

Eine Technologie die auch unsere Identitätsnarrative wohl am meisten verändert hat ist das Web.
Wie sich mit einem Klick eine Aussage über die eigene Identität treffen lässt sieht man beispielsweise auf Twitter oder Facebook – *klick* wir sind \enquote{Freunde}.
Ein Beispiel für den grad der Automatisierung mit der man diese Daten auch auswerten kann bieten Studien über Nutzerbeziehungen in Sozialen Netzwerken wie die Twitterbeziehungen der Abgeordneten des Europaparlamentes\parencite{maireder}.

\subsection {Fazit}

Ich kann festhalten, dass Automatisierung einen Effekt auf unserer Identität hat, der tatsächlich über einen Veränderung der Inhalte hinausgeht.
Der Grund für diese Veränderung liegt hauptsächlich in der für uns steigenden komplexität des Alltages, die schlicht daher rührt, dass die einfachen Tätigkeiten heute von Maschinen erledigt werden, die zu überwachen wiederum eine komplexe Aufgabe ist.
Allerdings ist diese Erklärung bei weitem nicht hinreichend um die in der Postmoderne entstandenen Identitätskonzepte zu erklären.
Vor allem die Änderungen in unserem Sozialen Umfeld hängen zwar vielleicht teilweise mit Automatisierung zusammen, sind aber in ihrer Ausprägung nicht durch Automatisierung zu erklären – womit Automatierung als Erkärungsansatz für diese Änderungen ausscheidet.

\newpage
\sloppy
\printbibliography
\include{chapters/erklaerung}
\end{document}
