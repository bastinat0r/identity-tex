\section {Automatisierte Identität}

Mitlerweile sind nicht nur die Möglichkeitsräume und Angebote unserer Identität von Automatisierung betroffen.
Das unsere Identität ist selbst durch Automatisierung betroffen ist, soll das folgende Beispiel zeigen.

\subsection {Mechanisierung und Automatisierung der Nationalidentität}

Stellen wir uns einmal vor wir stehen an der Grenze und möchten in ein anderes Land einreisen.
Normalerweise sehen wir dort einen Grenzbeamten, der wissen möchte wer wir sind.
Die Technik mit dem wir diesen Prozess sehr einfach machen ist der Pass, man könnte sagen das vorzeigen des Passes ist eine mechanisierte Form unserer Identifizierung.
Allerdings ist heute schon absehbar, dass das vorzeigen des Passes und die Arbeit des Grenzbeamten der unsere Identität feststellt bald der Vergangenheit angehören könnte.
Die Technologie der Biometrie, also der automatischen Identitätsfeststellung übernimmt diese Aufgabe \parencite{knaut}.

\subsection {Automatisierung der Identitätsarbeit im Internet}

Eine Technologie die auch unsere Identitätsnarrative wohl am meisten verändert hat ist das Web.
Wie sich mit einem Klick eine Aussage über die eigene Identität treffen lässt sieht man beispielsweise auf Twitter oder Facebook – *klick* wir sind \enquote{Freunde}.
Ein Beispiel für den grad der Automatisierung mit der man diese Daten auch auswerten kann bieten Studien über Nutzerbeziehungen in Sozialen Netzwerken wie die von \cite{maireder}.

\subsection {Fazit}

Ich kann festhalten, dass Automatisierung einen Effekt auf unserer Identität hat, der tatsächlich über einen Veränderung der Inhalte hinausgeht.
Der Grund für diese Veränderung liegt hauptsächlich in der für uns steigenden komplexität des Alltages, die schlicht daher rührt, dass die einfachen Tätigkeiten heute von Maschinen erledigt werden, die zu überwachen wiederum eine komplexe Aufgabe ist.
