\section{Postmoderne Identitätskonzepte und Automatisierung}

Da ich längst nicht der erste bin, der sich mit Identität in der Postmoderne Beschäftigt, möchte ich nicht von Null auf anfangen ein neues, auf Automatisierung beruhendes Identitätskonzept zu beschreiben, sondern bestehende Identitätskonzepte der Postmoderne aus dem Blickwinkel der Automatisierung und Mechanisierung beschreiben.

Alle drei Identitätskonzepte lassen sich mit der These vereinbaren, dass Automatisierung 

\subsection{Beschleunigung und Situative Identität}

Hartmuth Rosa führt als westentliche Ursache der Kriese von IdentitätskonzeptenVeränderungen in der Temporalstruktur an \parencite{rosa}.
Es ist durchaus plausibel, dass diese Veränderungen zumindest Teilweise in der vortschreitenden Mechanisierung und Automatisierung begründet liegen.
Der Automatisierungsprozess führt schlicht zu einer Verkürzung unserer Handlungen, die dann in schnellerer Abfolge durchgeführt werden.
Auch der technische Fortschritt wäre nicht möglich, ohne neuen Produkte in kurzer Zeit und großer Stückzahl fertigen zu können.
Dieser zweite Aspekt der Beschleunigung ist also, wenn überhaupt eher eine Indirekte Folge der Automatisierung.
Die Beschleunigung des Sozialen Wandels lässt sich noch weniger als Folge von Automatisierung verargumentieren.

\subsection{Flexible Identität}

\subsection{Narrative Identität}
