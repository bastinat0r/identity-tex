\section{Postmoderne Identitätskonzepte und Automatisierung}

Da ich längst nicht der erste bin, der sich mit Identität in der Postmoderne Beschäftigt, möchte ich nicht von Null auf anfangen ein neues, auf Automatisierung beruhendes Identitätskonzept zu beschreiben, sondern bestehende Identitätskonzepte der Postmoderne aus dem Blickwinkel der Automatisierung und Mechanisierung beschreiben.

Als Effekt dieser Veränderung lässt sich vorwegnehmen: Die Kontinuität der einzelnen Identitäten schwindet immer mehr und obwohl die Möglichkeiten für den einzelnen de facto immer mehr werden, sinkt das Kontingenzbewusstsein.
Außerdem lässt sich feststellen, dass Individualisierung schwieriger wird, da zwar die Ausdrucksmöglichkeiten für den Einzelnen immer größer werden, aber es mit zuhnehmender Pluralisierung der Lebensformen auch immer schwieriger wird, aus der Masse herauszustechen.

\todo{Kontingenzbewusstsein, Individualisierung, Kontinuität}

\subsection{Beschleunigung und Situative Identität}

Hartmuth Rosa führt als westentliche Ursache der Kriese von IdentitätskonzeptenVeränderungen in der Temporalstruktur an \parencite{rosa}.
Es ist durchaus plausibel, dass diese Veränderungen zumindest Teilweise in der vortschreitenden Mechanisierung und Automatisierung begründet liegen.
Diese Beschleunigung spiegelt sich in drei Aspekten wieder: Technische Beschleunigung, Beschleunigung des individuellen Lebenstempos und des sozialen Wandels.

Die Technische Beschleunigung von der Rosa spricht ist genau die, die aus Mechanisierung und Automatisierung hervorgeht – eine Reise die früher Tage gedauert hätte, dauert heute nur wenige Stunden, wo man früher einen Brief geschrieben hätte, schreibt man heute eine Mail.
Nicht unabhängig davon ist die Beschleunigung des individuellen Lebenstempos.
Paradoxer Weise führt die Einsparung von Zeit nicht dazu, dass uns mehr Zeit zur Verfügung steht – ganz im Gegenteil: Die Verkürzung einzelner Handlungsepisoden führt zu einer weiteren Steigerung des Lebenstempos, da mehr Handlungen in kürzerer Zeit ausgeführt werden\footnote{Genau das wird auch in der Debatte um eine \enquote{Automatisierungsdividende} thematisiert, also der Überlegung die versucht die durch Automatisierung freigewordenen Ressourcen für einzelne Personen zu nutzen, anstatt weiter nach Vollbeschäftigung mit langen Arbeitszeiten zu streben\parencite{faz}.}.
Es die einzelnen Handlungsepisoden sind also verkürzt, gleichzeitig steigt die Handlungsdichte.

Die dritte Art der Beschleunigung – die des Sozialen Wandels – lässt sich nicht auf Automatisierung oder Mechanisierung zurückführen.
Sich verändernde soziale Praktiken, Werte oder Traditionen ändern sich zwar möglicherweise weil sich unsere Umgebung durch technische Entwicklung ändert, die enstehende Struktur ist aber sicher nicht auf diese abstrakte Art von Veränderung zurückzuführen.
In den Änderungen der Sozialen Strukturen, liegt auch ein ganz deutliches Problem der These, dass Automatisierung die Struktur von Identität beiträgt, denn Soziale Änderungen lassen sich kaum auf Automatisierung zurückführen\footnote{Technologie gibt zwar auch Sozialstrukturen vor (beispielsweise auf Twitter, wo durch die Festlegung wer wann mit wem kommunuzieren darf, genau festgelegt ist}.




\todo{sebstthematisierung und artikulationsnot}

\subsection{Flexible Identität}



\subsection{Narrative Identität}
