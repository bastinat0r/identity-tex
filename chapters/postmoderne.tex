\section{Postmoderne Identitätskonzepte und Automatisierung}

Da ich längst nicht der erste bin, der sich mit Identität in der Postmoderne Beschäftigt, möchte ich nicht von Null auf anfangen ein neues, auf Automatisierung beruhendes Identitätskonzept zu beschreiben, sondern bestehende Identitätskonzepte der Postmoderne aus dem Blickwinkel der Automatisierung und Mechanisierung beschreiben.

Als Effekt dieser Veränderung lässt sich vorwegnehmen: Die Kontinuität der einzelnen Identitäten schwindet immer mehr und obwohl die Möglichkeiten für den einzelnen de facto immer mehr werden, sinkt das Kontingenzbewusstsein.
Außerdem lässt sich feststellen, dass Individualisierung schwieriger wird, da zwar die Ausdrucksmöglichkeiten für den Einzelnen immer größer werden, aber es mit zuhnehmender Pluralisierung der Lebensformen auch immer schwieriger wird, aus der Masse herauszustechen.

\todo{Kontingenzbewusstsein, Individualisierung, Kontinuität}

\subsection{Beschleunigung}

Hartmuth Rosa führt als westentliche Ursache der Kriese von IdentitätskonzeptenVeränderungen in der Temporalstruktur an \parencite{rosa}.
Es ist durchaus plausibel, dass diese Veränderungen zumindest Teilweise in der vortschreitenden Mechanisierung und Automatisierung begründet liegen.
Diese Beschleunigung spiegelt sich in drei Aspekten wieder: Technische Beschleunigung, Beschleunigung des individuellen Lebenstempos und des sozialen Wandels.

Die Technische Beschleunigung von der Rosa spricht ist genau die, die aus Mechanisierung und Automatisierung hervorgeht – eine Reise die früher Tage gedauert hätte, dauert heute nur wenige Stunden, wo man früher einen Brief geschrieben hätte, schreibt man heute eine Mail.
Nicht unabhängig davon ist die Beschleunigung des individuellen Lebenstempos.
Paradoxer Weise führt die Einsparung von Zeit nicht dazu, dass uns mehr Zeit zur Verfügung steht – ganz im Gegenteil: Die Verkürzung einzelner Handlungsepisoden führt zu einer weiteren Steigerung des Lebenstempos, da mehr Handlungen in kürzerer Zeit ausgeführt werden\footnote{Genau das wird auch in der Debatte um eine \enquote{Automatisierungsdividende} thematisiert, also der Überlegung die versucht die durch Automatisierung freigewordenen Ressourcen für einzelne Personen zu nutzen, anstatt weiter nach Vollbeschäftigung mit langen Arbeitszeiten zu streben\parencite{faz}.}.
Es die einzelnen Handlungsepisoden sind also verkürzt, gleichzeitig steigt die Handlungsdichte.

Die dritte Art der Beschleunigung – die des Sozialen Wandels – lässt sich nicht auf Automatisierung oder Mechanisierung zurückführen.
Sich verändernde soziale Praktiken, Werte oder Traditionen ändern sich zwar möglicherweise weil sich unsere Umgebung durch technische Entwicklung ändert,
der Soziale Wandel steht also in Wechselwirkung mit dem Technischen\enquote{rosa1}, die enstehende Struktur ist allerdings nicht auf diese abstrakte Art von Veränderung zurückzuführen.
\todo{den nächsten satz nochmal neu bitte}
In den Änderungen der Sozialen Strukturen, liegt auch ein ganz deutliches Problem der These, dass Automatisierung die Struktur von Identität beiträgt, denn Soziale Änderungen lassen sich kaum auf Automatisierung zurückführen\footnote{Technologie gibt zwar auch Sozialstrukturen vor (beispielsweise auf Twitter, wo durch die Festlegung wer wann mit wem kommunuzieren darf) allerindgs sind diese Neuerungen nicht strukturell von Automatisierungseigenschaften abhängig}.

Wie wirkt sich nun die Beschleunigung auf unsere Identität aus? Aufgrund der Vielzahl an Optionen, die unsere Indentitätsvorstellungen Betreffen kommt es zu einer Artikulationsnot, die nicht zuletzt in einem gesteigertem Kontingenzbewusstsein begründet liegt.
All die zum Teil technischen Möglichkeiten alles mögliche zu tun – und am nächsten Tag vielleicht etwas komplett neues zu tun, berauben uns des Vokabulars unsere Identitätsvorstellungen in Worte zu fassen und unsere Entscheidungen in Abgrenzung zu all den anderen Identitätsangeboten zu begründen.
Wo früher die Aussage \enquote{Ich möchte Mathematiker werden} eine klare Karriere vorgegeben hätte, wird man heute fragen, welche Richtung der Mathematik denn gemeint sei und auf welcher Laufbahn man dort hinzukommen gedenkt.
Und das alles in der Gewissheit, dass man mit abgeschlossenem Mathemitkstudium vielleicht am Ende trotzdem als Programmierer arbeitet.

Neben dieser Artikulationsnot gibt es allerdings einen Zwang zur Selbstthematisierung – denn ohne uns selbst ständig zum Thema zu machen können wir uns in den gewachsenen Möglichkeitsräumen nicht mehr zurechtfinden.
Auch die gestiegene Geschwindigkeit und zeitliche Dichte der Kommunikation trägt dazu bei.
Genau diese Änderung im Kommunikationsverhalten lässt sich aber wiederum auf allgemeine Eigenschaften der Mechanisierung zurückführen.
Zu aller erst werden die wenig komplexen Kommunikationsformen durch technische Werkzeuge ersetzt: Massenmedien wie Zeitungen, Radio oder die Kommunikation per Post. Jetzt allerdings sehen wir neue Werkzeuge mit denen sich komplexere Kommunikationsformen abbilden lassen: Telefonkonferenzen, Gruppenchats, Mailinglisten, Webforen – alles Kommunikationsformen bei denen es nichtmehr zwei Kommunikationspartner oder einen Autor der im wesentlichen unidirektional Kommuniziert gibt, sondern eine multidirektionale Kommunikation in einem Netzwerk zulässt.

Die Diagnose lautet nun, dass der Identität die Kontinuität zu einem guten Teil verloren geht und in jeder Situation neu ausgehandelt wird.
Ausgangspunkt ist dabei ein Optionsraum, der nicht unähnlich zu Bourdieus Sozialem Raum habituell beschränkt ist.
\todo{sagen, was das jetzt wieder mit automatisierung zu tun hat}


\subsection{Flexibilität}

Richard Sennet führt als Grund für die Änderungen unserer Berufe die zunehemende Flexibilität an.
Die Routine verschwindet mehr und mehr aus unserer Arbeitsrealität und wird ersetzt durch immer neue und nicht endende Veränderungen.
Wiederkehrende Aufgaben, die in den meisten Jobs zur \enquote{Routine} gehören, wie etwa das bestellen neuer Teile oder der Steuerung von Lager und Logistik wird zunehmend von Computern oder Maschinen erledigt – übrig bleiben die Spezialfälle und Störungen, also gerade die Abweichungen von dieser Routine, die zu bearbeiten uns wiederum ein breites Spektrum von Fähigkeiten und eben Flexibilität\footnote{Auch hier ist eines der auftretenden Probleme die Komplexität der Prozesse, die nicht zu selten von der Maschine (die nur ein bestimmtes Maß an Komplexität bewältigen kann) an uns (die wir flexibel auf diese Komplexen Situationen Regaieren können) weitergegeben wird} abverlangt.
Mann kann also durchaus behaupten, dass Automatisierung genau diese Steigerung der uns abverlangten Flexibilität zur Folge hat.

Betrachtet man die Umstrukturierung von Unternehmen, so wird man feststellen, dass auch hier die Flexibilität in den Vordergund rückt und flache Hierarchien die ehemaligen Strukturen ersetzen.
Eine Änderung wird während dieser Umstrukturierungen in den Unternehmen meist zusätzlich gemacht: Die Einführung Software-Management-Systeme wie von SAP oder Oracle.

\begin{quote}
	In Großkonzernen, die typischerweise Beschaffund und Einkauf, Inentarisierung, Budgetplanung, Personalverwaltung, Produkionsplanung, Kundenbeziehungsmanagement und Orderverwaltung in riesige Softwarepakete ausgelagert haben, wie sie der Waldorfer Konzern SAP herstellt [...] wird nicht einfach Software installiert. Damit die Optimierungspotenziale möglichst weit ausgeschöpft werden können, ist man dazu übergegangen, bei dieser Gelegenheit auch die vollständigen Geschäftsprozesse umzubauen, die künftig von Software gesteuert und betrieben werden. Diese Prozesse müssen schließlich nicht mehr primär an den Bedürfnissen und Fähigkeiten der Menschen orientiert sein, sondern an den Notwendigkeiten einer möglichst weitgehenden Digitalisierung und algorithmischen Verarbeitbarkeit, sowie an den Restriktionen der dominierenden Software.
	
	\em \cite[172f.]{arbeitsfrei}
\end{quote}

Es ist also sehr deutlich, dass die Einführung dieser Flachen Strukturen klein bloßer Zufall sind, sondern durchaus strukturell aus den Bedingungen der Automatisierung folgen\footnote{Inweiweit die Ausprägung dieser Technologie auch anders funktionieren würde, kann nur Spekulation bleiben, da der Mark für solche Softwaresysteme recht überschaubar ist\parencite[173]{arbeitsfrei}.}.

Interessant ist, welche Machtstrukturen aus diesen neuen Organisationsformen hervorgehen.
Die vermeintliche Freiheit der flexiblen Arbeitsformen stellt sich schnell als Wolf im Schafspelz dar: Wer heute flexible Arbeitszeiten hat und damit auch am Wochenende arbeiten darf, wird nicht selten auf einmal in der Situation sein, dass er tatsächlich auch am Wochenende Arbeiten muss. Und nicht selten Arbeiten Menschen die nicht nach Zeit bezahlt werden schlichtweg länger, weil Meilensteine und Termine gehalten werden müssen – und nach deren Erreichen normal weitergearbeitet wird, anstelle die \enquote{Überstunden wieder abzusetzen}.
Das Machtverhältnis entsteht dabei aus einer Beziehung der \enquote{Konzentration ohne Zentralisierung} und entsteht nicht aus einer Hierarchie, sondern aus der Organisationsstruktur heraus\footnote{Ich bin mir fast sicher, dass man auch in diesen Organisationsstrukturen die Merkmale des \enquote{Panoptischen Prinzips} feststellen könnte \parencite{foucault}.} \parencite[70]{sennett}.
Auch hier ist also die Frage, ob die Software für uns oder wir (irgendwann einmal?) für die Software arbeiten – und damit im \enquote{Stählrenen Gehäuse} enden.

Was ist unsere Motivation, diese Arbeitsverhältnisse hinzunehmen?
Hier kann auf jeden Fall die \enquote{Angst vor dem Scheitern} angeführt werden\parencite[159ff.]{sennett}, die nicht unähnlich zur diffusen Angst des Außengeleiteten Menschen wirkt, oder das Gegenstück dieser Angst: Der Wille \enquote{es zu etwas zu bringen}.
Es geht also nicht zuletzt darum, der steigenden Komplexität Herr zu werden, die einerseits an sich zunimmt, andererseits aber auch dadurch verstärkt wird, dass immer neue Komplexe Systeme verstanden und bewältiget werden müssen.


\subsection{Narrative}

Die wissenschaftliche Betrachtung der Identität findet auch im Bereich der Naratologie statt.
Eine Neuerung dieser Betrachtung ist, dass die Komplexität der Narrative vollständige Beachtung findet, und die Widersprüche nicht mehr zugunsten der Kohärenz zu reduzieren\parencite[164]{kraus}.
Zusätzlich zu dieser Änderung der wissenschaftlichen Praxis steigt auch die Komplexität der erzählten Geschichten.
Kommen wir noch einmal zurück zur Artikulationsnot. Auch ein Teil dieser Artikulationsnot dürfe seinen Ursprung in der durch Automatisierung und Mechanisierung gestiegenen Komplexität haben.
Ein Tuchwebermeister vor der Industriellen Revolution beantwortet die Frage \enquote{Was machst du eigentlich?} vermutlich relativ einfach und nachvollziehbar.
Der Besitzer einer Textilfabrik muss da schon weiter ausholen und komplizierte Maschinen beschreiben.
Unsere Narrative sind also ganz wesentlich beeinflusst von der gestiegenen Beschreibungskomplexität\footnote{Eine gestiegene Beschreibungskomplexität bedeutet, dass die Beschreibung einer Tätigkeit notwendigerweise komplexer wird.} unserer Tätigkeiten.

Zusätzlich bieten sich auch gänzlich neue Narrative.
Wo früher tatsächlich nur Erzählungen in Sprachform identitätsstiftend sein konnten, haben wir heute wesentlich mehr Technologien, die uns bei der Narration helfen können.
Beispielsweise kann man mit der Kamera ein Telefons ein \enquote{Selfie} machen und über Soziale Netzwerke teilen – ein kommunikationsvorgang der Definitiv als Identitätsnarration interpretiert werden kann\parencite{iqani}.
Es ergeben sich also durchaus neue Angebote der Identitätsnarration, die auch von Mechanisierung und Automatisierung profitieren, speziell geht es dabei aber ausschließlich um die Mechanasierung und Automatisierung von Kommunikationsprozessen.
Einen anderen solchen Narrativ hat die \enquote{Quantified-Self Bewegung} für sich entdeckt\parencite{bellinger}.
Die (automatisierte) Erhebung von Daten und die Möglichkeit diese zu veröffentlichen bieten eine gänzlich neue Form der Narration, die obendrein automatisch ablaufen kann.
Man kann also durchaus behaupten, dass die Narrationsarbeit selbst automatisiert wird.


