\section {Mechanisierung, Automatisierung und Identität}

Unter dem Umgangssprachlich gebrauchten Begriff \enquote{Automatisierung} verbergen sich eigentlich zwei voneinander Abgrenzbare Begriffe: Mechanisierung und Automatisierung (im eigentlichen Sinne).

\subsection {Mechanisierung}

Als Mechanisierung Bezeichen ich in dieser Arbeit den Einsatz von Techniken, die menschliche Kraft unterstützen oder ersetzen, und so den Aufwand für unsere Handlungen senken, oder deren Etrag steigern.
Das durchführen der eigentlichen Handlung ist dabei aber immer noch notwendig.

Ein Rechenschieber und Rechentabellen beispielsweise machen es möglich, dass ich mit kleinerem Aufwand Berechnungen durchführen kann.
Der Einsatz des Rechenschiebers ist also als Mechanisierung zu verstehen.

\subsection {Automatisierung}

Automatisierung wird für die Industrie definiert als: \enquote{Das Ausrüsten einer Einrichtung, so dass sie ganz oder teilweise ohne Mitwirkung des Menschen bestimmungsgemäß arbeitet.}\parencite{din19233}.

Die Automatisierung kann als Weiterführung der Mechanisierung verstanden werden\parencite{ulrich}, ist allerdings dadurch ausgezeichnet, dass der Mensch nicht mehr Steuernd in Prozesse eingreifen muss und alle Zwischenschritte durch die Maschine erledigt werden.
Wichtig für die Abgrenzung zur Mechanisierung ist, dass die ursprüngliche Handlung im automatisierten Vorgang gar nicht mehr durchgeführt wird – während durch eine Mechanisierung nur verändert wird.


Führen wir das Beispiel des Rechnens weiter, so ist die Automatisierung des Rechnens der Einsatz eines Computers, der seine Eingaben selbständig abruft und auch selbstständig ein Ergebnis errechnet.

\subsection {Das Verhältnis von Mechanisierung und Automatisierung zur Identität}

Die Abgrenzung zwischen beiden Prozessen ist oft nicht eindeutig und der Übergang von Mechanisierung zur Automatisierung oft fließend, allerdings spielt sich an eben diesem Übergang ein wichtiger Wandel für die Identität des Betroffenen ab. So kann ich mich als Mensch mit Rechenschieber, Zettel und Stift noch \enquote{Computer} nennen heutzutage wo das eigentliche Rechen aber vollständig automatisch abläuft geht das nicht mehr. Ich bin vielleicht \enquote{Systemadministrator} oder \enquote{Programmierer}, aber die eigentliche Tätigkeit des Rechnens gehört jetz nicht mehr zu meiner Tätigkeit – und ist damit auch nicht mehr identitätsstiftend.

Den technischen Prozess zu überwachen ist in gewissen weise eine logische Nachfolge der eigentlichen Tätigkeit, allerdings kann es für die Identitätsbildung nicht das leisten, was die ursprüngliche Tätigkeit leisten kann.
Die Arbeit wird abstrakter und bekommt meist einen weiteren Themenumfang als es ursprünglich der Fall war \parencite{ulrich} – Automatisiert und Mechanisiert werden zunächst mechanisch einfache Vorgänge von niedrigem Komplexitätsgrad, die häufig wiederholt werden.
Zusätzlich zur Verlagerung des Arbeitsfeldes ist auch der durch die Automatisierung erreichte Rationalisierungseffekt wichtig für Identitätskonzepte.
Von den 100 Bauern die vor 100 Jahren noch auf einem Bauernhof gearbeitet haben, sind heute gerade noch anderthalb Stellen für sehr gut ausgebildete Landwirte übriggeblieben \parencite{arbeitsfrei} – die übrigen 98,5 Menschen verlieren die Identitätsstiftenden tätigkeiten auf dem Bauernhof komplett.

Da heutzutage alle Lebensbereiche in der ein oder anderen Weise von Automatisierung betroffen sind, ist anzunehmen, dass sich usere Identitätskonzepte zwangsläufig ändern müssen.

