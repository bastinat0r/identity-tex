\section {Mechanisierung, Automatisierung und Identität}

Unter dem Umgangssprachlich gebrauchten Begriff \enquote{Automatisierung} verbergen sich eigentlich zwei voneinander Abgrenzbare Begriffe: Mechanisierung und Automatisierung (im eigentlichen Sinne).

\subsection {Mechanisierung}

Als Mechanisierung Bezeichen ich in dieser Arbeit den Einsatz von Techniken, die menschliche Kraft unterstützen oder ersetzen, und so den Aufwand für unsere Handlungen senken, oder deren Etrag steigern.
Das durchführen der eigentlichen Handlung ist dabei aber immer noch notwendig.

Ein Rechenschieber und Rechentabellen beispielsweise machen es möglich, dass ich mit kleinerem Aufwand Berechnungen durchführen kann.
Der Einsatz des Rechenschiebers ist also als Mechanisierung zu verstehen.

\subsection {Automatisierung}

Automatisierung wird für die Industrie definiert als: \enquote{Das Ausrüsten einer Einrichtung, so dass sie ganz oder teilweise ohne Mitwirkung des Menschen bestimmungsgemäß arbeitet.}\parencite{din19233}.

Die Automatisierung kann als Weiterführung der Mechanisierung verstanden werden\parencite{ulrich}, ist allerdings dadurch ausgezeichnet, dass der Mensch nicht mehr Steuernd in Prozesse eingreifen muss und alle Zwischenschritte durch die Maschine erledigt werden.
Wichtig für die Abgrenzung zur Mechanisierung ist, dass die ursprüngliche Handlung im automatisierten Vorgang gar nicht mehr durchgeführt wird – während durch eine Mechanisierung nur verändert wird.


Führen wir das Beispiel des Rechnens weiter, so ist die Automatisierung des Rechnens der Einsatz eines Computers, der seine Eingaben selbständig abruft und auch selbstständig ein Ergebnis errechnet.

\subsection {Identität}
Identität verstehe ich zunächst als die Antwort auf die Frage: \enquote{Wer bin ich?}.

Die Fragen die dabei mitschwingen sind \enquote{Wer war ich?} und \enquote{Was sind meine Ziele?} – also eine Perspektive die einerseits unsere Vergangenheit berücksichtigt und andererseits Möglichkeitsräume einbezieht.
Neben dieser zeitlichen Perspektive werden bei der Identitätsbildung Verknüpfungsarbeit geleistet zwischen lebensweltlichen und inhaltlichen Perspektiven.
Das aushandeln der Relevanz der einzelnen Tatsachen ist zentraler Bestndteil der Identitätsarbeit.
Die Konflikte, die sich dabei aus den verschiedenen Identitätsbestandteilen \footnote{self, me, other, ideal ...} ergeben, sind Quelle der Dynamik des Identitätsprozesses \parencite{keupp}.
Zusätzlich möchte ich den Narrativen Charakter der Identität betonen.
Wenn Identität die Antwort auf eine Frage ist, so muss sie auch den gängigen Metanarrativen folgen

\subsection {Das Verhältnis von Mechanisierung und Automatisierung zur Identität}

Diese Möglichkeitsräume werden stark beeinflusst von Mechanisierung und Automatisierung.
Einerseits eröffnen sich neue Möglichkeiten, da Mechanisierung schlichtweg unsere Effizienz und Effektivität steigern, andererseits verschwinden auch Möglichkeiten für unser Handeln, zum Beispiel weil alte Berufe nicht mehr ausgeübt werden.

Die Abgrenzung zwischen beiden Prozessen der Mechanisierung und Automatisierung ist oft nicht eindeutig und der Übergang von Mechanisierung zur Automatisierung oft fließend, allerdings ist dieser Übergang wichtig für die Identität derer die davon Betroffen sind.
So kann ich mich als Mensch mit Rechenschieber, Zettel und Stift noch \enquote{Computer}\footnote{\enquote{Computer} war früher tatsächlich eine Berufsbezeichnung. Das wir heute  eine Maschine, aber niemals einen Menschen so bezeichnen würden zeigt, wie überholt dieses Identitätsangebot ist.} nennen heutzutage wo das eigentliche Rechen aber vollständig automatisch abläuft geht das nicht mehr.
Ich bin vielleicht \enquote{Systemadministrator} oder \enquote{Programmierer}, aber die eigentliche Tätigkeit des Rechnens gehört jetz nicht mehr zu meiner Tätigkeit – und ist damit auch nicht mehr identitätsstiftend.

Den technischen Prozess zu überwachen ist in gewissen weise eine logische Nachfolge der eigentlichen Tätigkeit, allerdings kann es für die Identitätsbildung nicht das leisten, was die ursprüngliche Tätigkeit leisten kann.
Die Arbeit wird abstrakter und bekommt meist einen weiteren Themenumfang als es ursprünglich der Fall war \parencite{ulrich} – Automatisiert und Mechanisiert werden zunächst mechanisch einfache Vorgänge von niedrigem Komplexitätsgrad, die häufig wiederholt werden.
Zusätzlich zur Verlagerung des Arbeitsfeldes ist auch der durch die Automatisierung erreichte Rationalisierungseffekt wichtig für Identitätskonzepte.
Von den 100 Bauern die vor 100 Jahren noch auf einem Bauernhof gearbeitet haben, sind heute gerade noch anderthalb Stellen für sehr gut ausgebildete Landwirte übriggeblieben \parencite{arbeitsfrei} – die übrigen 98,5 Menschen verlieren die Identitätsstiftenden Tätigkeiten auf dem Bauernhof komplett.

Da heutzutage alle Lebensbereiche in der ein oder anderen Weise von Automatisierung betroffen sind, ist anzunehmen, dass sich usere Identitäten zwangsläufig ändern müssen.
Die Frage die ich zu beantworten versuche ist: Ist diese Veränderung von struktureller oder inhaltlicher Natur.

