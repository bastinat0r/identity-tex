\section {Mechanisierung, Automatisierung und Identität}

Unter dem Umgangssprachlich gebrauchten Begriff \enquote{Automatisierung} verbergen sich eigentlich zwei voneinander Abgrenzbare Begriffe: Mechanisierung und Automatisierung (im eigentlichen Sinne).

\subsection {Mechanisierung}

Als Mechanisierung Bezeichen ich in dieser Arbeit den Einsatz von Techniken, die menschliche Kraft unterstützen oder ersetzen, und so den Aufwand für unsere Handlungen senken, oder deren Etrag steigern. Das durchführen der eigentlichen Handlung ist dabei aber immer noch notwendig. Ein Rechenschieber und Rechentabellen beispielsweise machen es möglich, dass ich mit kleinerem Aufwand Berechnungen durchführen kann.
Der Einsatz des Rechenschiebers ist also als Mechanisierung zu verstehen.

\subsection {Automatisierung}

Automatisierung wird für die Industrie definiert als: \enquote{Das Ausrüsten einer Einrichtung, so dass sie ganz oder teilweise ohne Mitwirkung des Menschen bestimmungsgemäß arbeitet.}\parencite{din19233}.

\subsection {Das Verhältnis von Mechanisierung und Automatisierung zur Identität}

Die Abgrenzung zwischen beiden Prozessen ist oft nicht eindeutig und der Übergang von Mechanisierung zur Automatisierung oft fließend, allerdings spielt sich an eben diesem Übergang ein wichtiger Wandel für die Identität des Betroffenen ab. So kann ich mich als Mensch mit Rechenschieber, Zettel und Stift noch \enquote{Computer} nennen heutzutage wo das eigentliche Rechen aber vollständig automatisch abläuft geht das nicht mehr. Ich bin vielleicht \enquote{Systemadministrator} oder \enquote{Programmierer}, aber die eigentliche Tätigkeit des Rechnens gehört jetz nicht mehr zu meiner Tätigkeit.

\subsection {Digitale Revolution, Computerisierung}
Mit der Erfindung und Miniaturisierung der Computer geht die Automatisierung in eine neue Phase ein. Wo früher fast nur Produktionsprozzesse automatisiert wurden, erfasst die Automatisierung heute jeden Lebensbereich.
Sogar unsere Kommunikation ist von automatisierung Betroffen – das \enquote{Das wollte ich gar nicht schreiben, das war die Autokorrektur auf meinem Telefon} heutzutage ein völlig normaler Narrativ ist zeigt, welches Außmaß Automatisierung heute annimmt; es ist auch nicht davon auszugehen, dass das Automatisierungslevel nicht noch sehr viel weiter steigen wird \parencite{arbeitsfrei}.
