\section {Mechanisierung, Automatisierung und Identität}

Automatisierung ist kein neues, aber ein aktuelles Thema.
Egal ob man von technischem Fortschritt im Allgemeinen, dem marktwirtschaftlichen Rationalisierungseffekt oder politischen Konstrukten zum Gesellschaftlichen Umgang mit Automatisierung, wie der \enquote{Automatisierungsdividende} redet: Automatisierung wird immer wichtiger für uns und immer präsenter in unseren Leben.
Die Frage die sich stellt ist: Ändert unser Einsatz dieser Technologien wer wir sind? Lassen sich allgemeine Veränderungen feststellen, die dieser abstrakte technologische Prozess in unserer Identität vornimmt?

Um diese Frage zu beantworten müssen wir zunächst genauer abstecken, was gemeint ist mit \enquote{Automatisierung}:
Unter dem Umgangssprachlich gebrauchten Begriff \enquote{Automatisierung} verbergen sich eigentlich zwei voneinander abgrenzbare Begriffe: Mechanisierung und Automatisierung (im eigentlichen Sinne). Die im Folgenden brauchbar definiert werden sollen.

\subsection {Mechanisierung}

Der eigentlichen Automatisierung geht historisch und auch in technologischen Entwicklungen normalerweise ein Schritt vonstatten, der als Mechanisierung bezeichnet werden kann.
Als Mechanisierung bezeiche ich in dieser Arbeit den Einsatz von Techniken, die menschliche Kraft unterstützen oder ersetzen, und so den Aufwand für unsere Handlungen senken, oder deren Etrag steigern.
Das durchführen der eigentlichen Handlung ist dabei aber immer noch notwendig.
Der Entwicklungsprozess von Automatisierungstechnologien geht dabei oft mehrere Zwischenschritte, die den Aufwand der eingesetzten Arbeit immer geringer machen, bis schließlich der gesamte Prozess autonom abläuft.
Eine entscheidende Eigenschaft dieses Mechanisierungsprozessses ist, dass zuerst simple Arbeitsschritte von geringer Komplexität durch den Einsatz von Werkzeugen erweitert werden und komplexe Arbeitsgänge erst nach und nach durch Maschinen erledigt werden, da Maschinen für diese Arbeitsshritte teurer wären, diese Komplexitätsproblematik setzt sich natürlich auch zur Automatisierung fort.


\subsection {Automatisierung}

Automatisierung wird für die Industrie definiert als: \enquote{Das Ausrüsten einer Einrichtung, so dass sie ganz oder teilweise ohne Mitwirkung des Menschen bestimmungsgemäß arbeitet.}\parencite{din19233}.

Die Automatisierung kann als Weiterführung der Mechanisierung verstanden werden\parencite{ulrich}, ist allerdings dadurch ausgezeichnet, dass der Mensch nicht mehr Steuernd in Prozesse eingreifen muss und alle Zwischenschritte durch die Maschine erledigt werden\footnote{Ein Interessantes Phänomen in diesem Zusammenhang ist sicher das Outsourcing von Arbeit. Ob eine Tätigkeit von einer Maschine, oder einem schlecht bezahlten Arbeiter am anderen Ende der Welt erledigt wird hat für uns fast den gleichen Effekt – solange nämlich, wie wir von diesem Arbeiter nichts mitbekommen. Inwiefern dieses Outsourcing als (Kultur-)Technik gewertet werden kann, kann und möchte ich hier nicht beatworten – allerdings ist die Frage an sich durchaus interessant.}.
Wichtig für die Abgrenzung zur Mechanisierung ist, dass die ursprüngliche Handlung im automatisierten Vorgang gar nicht mehr durchgeführt wird – während durch eine Mechanisierung nur verändert wird.

Ein Rechenschieber und Rechentabellen machen es möglich, dass ich mit kleinerem Aufwand Berechnungen durchführen kann.
Der Einsatz des Rechenschiebers ist also als Mechanisierung zu verstehen.
Führen wir das Beispiel des Rechnens weiter, so ist die Automatisierung des Rechnens der Einsatz eines Computers, der seine Eingaben selbständig abruft und auch selbstständig ein Ergebnis errechnet – die Frage ab wann genau man von Automatisierung sprechen kann ist allerdings schon hier nicht leicht zu beantworten.

\subsection {Identität}
Identität verstehe ich zunächst als die Antwort auf die Frage: \enquote{Wer bin ich?}.

Die Fragen die dabei mitschwingen sind \enquote{Wer war ich?} und \enquote{Was sind meine Ziele?} – also eine Perspektive die einerseits unsere Vergangenheit berücksichtigt und andererseits Möglichkeitsräume einbezieht.
Neben dieser zeitlichen Perspektive werden bei der Identitätsbildung Verknüpfungsarbeit geleistet zwischen lebensweltlichen und inhaltlichen Perspektiven.
Das aushandeln der Relevanz der einzelnen Tatsachen ist zentraler Bestndteil der Identitätsarbeit.
Die Konflikte, die sich dabei aus den verschiedenen Identitätsbestandteilen \footnote{self, me, other, ideal ...} ergeben, sind Quelle der Dynamik des Identitätsprozesses \parencite{keupp}.
Zusätzlich möchte ich den Narrativen Charakter der Identität betonen:
Wenn Identität die Antwort auf eine Frage ist, so muss sie auch den gängigen Metanarrativen folgen und unterliegt damit zusätzlichen Einschränkungen und Bedingungen.

\subsection {Das Verhältnis von Mechanisierung und Automatisierung zur Identität}

Diese Möglichkeitsräume werden stark beeinflusst von Mechanisierung und Automatisierung.
Einerseits eröffnen sich neue Möglichkeiten, da Mechanisierung schlichtweg unsere Effizienz und Effektivität steigern, andererseits verschwinden auch Möglichkeiten für unser Handeln, zum Beispiel weil alte Berufe nicht mehr ausgeübt werden.

Die Abgrenzung zwischen beiden Prozessen der Mechanisierung und Automatisierung ist oft nicht eindeutig und der Übergang von Mechanisierung zur Automatisierung oft fließend, allerdings ist dieser Übergang wichtig für die Identität derer die davon Betroffen sind.
So kann ich mich als Mensch mit Rechenschieber, Zettel und Stift noch \enquote{Computer}\footnote{\enquote{Computer} war früher tatsächlich eine Berufsbezeichnung. Das wir heute  eine Maschine, aber niemals einen Menschen so bezeichnen würden zeigt, wie überholt dieses Identitätsangebot ist.} nennen heutzutage wo das eigentliche Rechen aber vollständig automatisch abläuft geht das nicht mehr.
Ich bin vielleicht \enquote{Systemadministrator} oder \enquote{Programmierer}, aber die eigentliche Tätigkeit des Rechnens gehört jetz nicht mehr zu meiner Tätigkeit – und ist damit auch nicht mehr identitätsstiftend.

Den technischen Prozess zu überwachen ist in gewissen weise eine logische Nachfolge der eigentlichen Tätigkeit, allerdings kann es für die Identitätsbildung nicht das leisten, was die ursprüngliche Tätigkeit leisten kann.
Die Arbeit wird abstrakter und bekommt meist einen weiteren Themenumfang als es ursprünglich der Fall war \parencite{ulrich} – Automatisiert und Mechanisiert werden zunächst mechanisch einfache Vorgänge von niedrigem Komplexitätsgrad, die häufig wiederholt werden.
Zusätzlich zur Verlagerung des Arbeitsfeldes ist auch der durch die Automatisierung erreichte Rationalisierungseffekt wichtig für Identitätskonzepte.
Von den 100 Bauern die vor 100 Jahren noch auf einem Bauernhof gearbeitet haben, sind heute gerade noch anderthalb Stellen für sehr gut ausgebildete Landwirte übriggeblieben \parencite{arbeitsfrei} – die übrigen 98,5 Menschen verlieren die Identitätsstiftenden Tätigkeiten auf dem Bauernhof komplett.

Da heutzutage alle Lebensbereiche in der ein oder anderen Weise von Automatisierung betroffen sind, ist anzunehmen, dass sich usere Identitäten zwangsläufig ändern müssen.
Die Frage die ich zu beantworten versuche ist: Ist diese Veränderung von struktureller oder ausschließlich inhaltlicher Natur?

