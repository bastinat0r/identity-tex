\section{Geschichte der Automatisierung}

Mechanisierung und Automatisierung gibt es seit der Mesch Werkzeuge benutzt, aber enorme Folgen für die Sozialstruktur beginnen in der Industriellen Revolution.

\subsection{Ausgangslage™}

Vor der Industriellen Revolution gab es nur eine Agrargesellschaft, die Frage \enquote{Wer bin ich?} lies sich relativ einfach beantworten mit \enquote{Ich bin Bauer [, Müller, Pfarrer, Schmied \footnote{Darin dürften sich die Rollenbilder der Zeit auch fast schon erschöpfen}] aus Kleinbuxtehude}.

Jeder tat das was er schon immer getan hat, die Menschen waren im wesentlichen von Tradion geleitet \parencite{riessman}.

\subsection{Industrielle Revolution}

Zum Beginn der Industrielllen Revolution haben wir es vor allem mit Mechanisierung zu tun.
Mit Entwicklung der Dampfmaschine wird dem Menschen (körperliche) Arbeit abgenommen, die jetzt von Maschinen erledigt wird \parencite{landes}.
Diese Revolution konnte überhaupt erst vonstatten gehen, da in der Landwirtschaft genügend Überschüsse erwirtschaftet wurden konnten, um Fabrikarbeiter zu ernähren. Diese Überschüsse wurden nicht aus purem Fleiß erarbeitet, sondern sind Folge einer Agrarrevolution, die man ebenfalls als Mechanisierung auffassen kann \parencite{weissenborn, prass}.

\subsection{Nach dem 2. Welkrieg – Automatisierung}


