\section{Geschichte der Automatisierung}

Mechanisierung und Automatisierung gibt es seit der Mesch Werkzeuge benutzt, aber enorme Folgen für die Sozialstruktur beginnen in der Industriellen Revolution.
Wenn wir \enquote{Müller} sagen, dann haben wir meist ein romantisches Bild in unserer Vorstellung, von einer staubigen alten Mühle mit Wind- oder Wasserrad in dem einige Müller ihr Tagewerk tun.
In einer Modernen Mühle dagegen arbeiten kaum Menschen \footnote{Vermutlich haben die Museumsmühlen insgesamt mehr Mitarbeiter als die Industriemühlen in denen unser Mehl hergestellt wird}, das Personal dort wird zum Überwachen und Instandhalten der Industrieanlagen gebraucht und nicht mehr zum mahlen von Getreide.
Das Bild des \enquote{Müllers} passt nicht einmal im Ansatz in die Postmoderne, und doch ist es das Identitätsbild, das wir im Kopf haben.

Welchen Kriesen sich diese Identität gegenüber sieht und wie sich unser Konzeptvon Identität seitdem geändert hat, möchte ich in diesem Kapitel erforschen.

\subsection{Ausgangslage™}

Vor der Industriellen Revolution gab es nur eine Agrargesellschaft, die Frage \enquote{Wer bin ich?} lies sich relativ einfach beantworten mit \enquote{Ich bin Bauer [, Müller, Pfarrer, Schmied \footnote{Darin dürften sich die Rollenbilder der Zeit auch fast schon erschöpfen}] aus Kleinbuxtehude}.

Jeder tat das was er oder seine Eltern schon immer getan hatten, die Menschen waren im wesentlichen von Tradion geleitet \parencite{riessman}.

\subsection{Industrielle Revolution}

Zum Beginn der Industrielllen Revolution haben wir es vor allem mit Mechanisierung zu tun.
Mit Entwicklung der Dampfmaschine wird dem Menschen (körperliche) Arbeit abgenommen, die jetzt von Maschinen erledigt wird \parencite{landes}.
Diese Revolution konnte überhaupt erst vonstatten gehen, da in der Landwirtschaft genügend Überschüsse erwirtschaftet wurden konnten, um Fabrikarbeiter zu ernähren. Diese Überschüsse wurden nicht aus purem Fleiß erarbeitet, sondern sind Folge einer Agrarrevolution, die man ebenfalls als Mechanisierung auffassen kann \parencite{weissenborn, prass}.

Nach Riesman ist hier der Idealtyp des Innengeleiteten Menschen vorherrschend. Der Innengeleitet Mensch wird durch bestimmte Werte geprägt denen er dann wie einer Kompassnadel folgt.
Riesman begründet den Wandel der Identitätskonzepte durch das Bevölkerungswachstum und die daraus resultierende Ressourcenknappheit, das wohl nicht zuletzt eine Folge der Agrarrevolution sein dürfte.


\subsection{Nach dem 2. Welkrieg – Automatisierung}

\subsection {Digitale Revolution, Computerisierung}
Mit der Erfindung und Miniaturisierung der Computer geht die Automatisierung in eine neue Phase ein. Wo früher fast nur Produktionsprozzesse automatisiert wurden, erfasst die Automatisierung heute jeden Lebensbereich.
Sogar unsere Kommunikation ist von automatisierung Betroffen – das \enquote{Das wollte ich gar nicht schreiben, das war die Autokorrektur auf meinem Telefon} heutzutage ein völlig normaler Narrativ ist zeigt, welches Außmaß Automatisierung heute annimmt; es ist auch nicht davon auszugehen, dass das Automatisierungslevel nicht noch sehr viel weiter steigen wird \parencite{arbeitsfrei}.

