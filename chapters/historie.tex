\section{Geschichte der Automatisierung}

Mechanisierung und Automatisierung gibt es seit der Mesch Werkzeuge benutzt, aber enorme Folgen für die Sozialstruktur beginnen in der Industriellen Revolution.
Wenn wir \enquote{Müller} sagen, dann haben wir meist ein romantisches Bild in unserer Vorstellung, von einer staubigen alten Mühle mit Wind- oder Wasserrad in dem einige Müller ihr Tagewerk tun.
In einer Modernen Mühle dagegen arbeiten kaum Menschen \footnote{Vermutlich haben die Museumsmühlen insgesamt mehr Mitarbeiter als die Industriemühlen in denen unser Mehl hergestellt wird}, das Personal dort wird zum Überwachen und Instandhalten der Industrieanlagen gebraucht und nicht mehr zum mahlen von Getreide.
Das Bild des \enquote{Müllers} passt nicht einmal im Ansatz in die Postmoderne, und doch ist es das Identitätsbild, das wir im Kopf haben.

Welchen Kriesen sich diese Identität gegenüber sieht und wie sich unser Konzeptvon Identität seitdem geändert hat, möchte ich in diesem Kapitel erforschen.

\todo{technologien der sozialen Sättigung}

\subsection{Ausgangslage™}

Vor der Industriellen Revolution gab es nur eine Agrargesellschaft, die Frage \enquote{Wer bin ich?} lies sich relativ einfach beantworten mit \enquote{Ich bin Bauer [, Müller, Pfarrer, Schmied \footnote{Darin dürften sich die Rollenbilder der Zeit auch fast schon erschöpfen}] aus Kleinbuxtehude}.


\subsection{Industrielle Revolution}

Zum Beginn der Industrielllen Revolution haben wir es vor allem mit Mechanisierung zu tun.
Mit Entwicklung der Dampfmaschine wird dem Menschen (körperliche) Arbeit abgenommen, die jetzt von Maschinen erledigt wird \parencite{landes}.
Diese Revolution konnte überhaupt erst vonstatten gehen, da in der Landwirtschaft genügend Überschüsse erwirtschaftet wurden konnten, um Fabrikarbeiter zu ernähren. Diese Überschüsse wurden nicht aus purem Fleiß erarbeitet, sondern sind Folge einer Agrarrevolution, deren Neuerungen man ebenfalls als Mechanisierung auffassen kann \parencite{weissenborn, prass}.

Schon in der industriellen Revolution gab es eine heftige Gegenbewegung, gegen die um sich greifende Mechanisierung: Die Maschinenstürmer.
Beispielsweise \enquote{zerstörten Wolltuchspinner und -weber, Kattundrucker und Tuchscherer neue [Textil-]Maschinen}\parencite[44]{spehr}.
Man kann schon an diesem Beispiel erkennen, dass es eine konkrete Identitätszuschreibung der handelnden Maschinenstürmer gibt (ihr Beruf in der Textilverarbeitung) und das sich diese Identität in der damaligen Form nicht aufrechterhalten ließ.


Diese Welle der Mechanisierung hört mit der Industriellen Revolution nicht auf.
Eine weiter Entwickung ist zum Beispiel die der Fließbandarbeit.
Interessanterweise führt hier die Mechanisierung des Transportes der Baugruppen durch die Fabrik nicht nur zu einer Arbeitsersparnis bei den jeweiligen Fertigungsprozessen, sondern auch die Komplexität der einzelnen Arbeitsschritte ist westentlich geringer als vorher – was eine weiter Mechanisierung und Automatisierung dieser Arbeitsschritte begünstigt.

\subsection {Digitale Revolution, Computerisierung}
Mit der Erfindung und Miniaturisierung der Computer geht die Automatisierung in eine neue Phase ein. Wo früher fast nur Produktionsprozzesse automatisiert wurden, erfasst die Automatisierung heute jeden Lebensbereich.
Wo vorher nur einfache Maschinen autonom agieren konnten machen neue Sensoren und deren rechnergestützte Auswertung Regelkreise möglich, die vorher wesentlich schwieriger zu realisieren waren.
Diese Regelkreise wiederum sind wichtiger Bestandteil autonom agierender Systeme \parencite{ulrich}.


Sogar unsere Kommunikation ist von automatisierung Betroffen – das \enquote{Das wollte ich gar nicht schreiben, das war die Autokorrektur auf meinem Telefon} heutzutage ein völlig normaler Narrativ ist zeigt, welches Außmaß Automatisierung heute annimmt; es ist auch nicht davon auszugehen, dass das Automatisierungslevel nicht noch sehr viel weiter steigen wird \parencite{arbeitsfrei}.

Wie weit die Automatisierung in Bereiche vorgedrungen ist, die man vor wenigen Jahren noch für komplett Automatisierungssicher hielt sieht man, wenn man sich ansieht wie Algorithmen einfache Berichte verfassen. Bereits 2005 wurden automatisch generierte Artikel zu Fachkonferenzen angenommen. Die Arbeiten die damals generiert wurden hatten noch keinen Sinnvollen Inhalti \parencite{scigen}.
Heute allerdings lassen sich auch komplexere Texte wie den Kommentar zu einem Fussballspiel oder die Zusammenfassung von Unternehmensdaten erstellen\parencite{bou...}.

\subsection{Identität}

Im Mittelalter, also vor Beginn der großen Mechanisierungswelle der Industriellen Revolution, waren die Menschen im wesentlichen von Tradion geleitet \parencite{riessman}.
Die Identität des einzelnen ist a priori festgelegt, meist von Geburt an\parencite{rosa}.

Nach Riesman ist zur Zeit der Industriellen Revolution der Idealtyp des Innengeleiteten Menschen vorherrschend. Der Innengeleitet Mensch wird durch bestimmte Werte geprägt denen er dann wie einer Kompassnadel folgt.
Die Identität wird als nicht mehr zur Geburt, sondern im Laufe des Lebens festgelegt, aber dann nicht mehr verändert.

Riesman begründet den Wandel der Identitätskonzepte durch das Bevölkerungs"-wachstum und die daraus resultierende Ressourcenknappheit, das wohl nicht zuletzt eine Folge der Agrarrevolution sein dürfte.
Versucht man die Änderungen komplett aus der Automatisierung und Mechanisierung zu begründen, so kann man anführen, dass die tradierte Verhaltensweise schlichtweg nicht mehr anwendbar waren. \todo{bullshit?}

Mit zunehmender rationalisierung in den Produktionsprozessen und einem einhalten des Bevölkerungswachstumes herrscht nun nicht mehr das Idealbild des Innengeleiteten, sondern das des außengeleiteten Menschen vor.
Die Identität wird nicht mehr zu einem speziellen Zeitpunkt festgelegt, sondern in jeder Situation neu ausgehandelt.
